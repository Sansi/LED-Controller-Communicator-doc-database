\section{表结构}\label{ux8868ux7ed3ux6784}

除特殊说明外,每张表的第一个字段为主键 (Primary Key)。

\subsection{SystemConfig}\label{systemconfig}

\begin{longtabu}[c]{@{}lXXX@{}}
\toprule
字段名 & 类型 & 含义 & 备注\tabularnewline
\midrule
\endhead
idSystemConfig & int unsigned AUTO\_INCREMENT & &\tabularnewline
NetworkMode & tinyint unsigned & StarRiver服务角色 & 0: 服务器, 1:
客户端。考虑同时作为多种角色与不同控制器通信的需求\tabularnewline
NetworkPort & smallint unsigned & 服务器侦听端口 &\tabularnewline
LastServerActivity & timestamp & StarRiver服务最近连接时间 &
StarRiver服务定时更新这个字段。\tabularnewline
IntervalReportActivity & int unsigned & 更新最近连接时间间隔
&\tabularnewline
IntervalAutoStatusQuery & int unsigned & 自动查询控制器状态间隔
&\tabularnewline
IntervalAutoDeviceStatusQuery & int unsigned & 自动查询设备状态间隔
&\tabularnewline
IntervalRetryWrite & int unsigned & 连接建立期间,写操作重试间隔 &
单位为毫秒\tabularnewline
IntervalKeepalive & int unsigned & 心跳包发送间隔 &\tabularnewline
TimeoutCmdAck & smallint unsigned & 指令响应超时 & CSA 6.1
T1\tabularnewline
TimeoutCmdResult & smallint unsigned & 指令结果超时 & CSA 6.1
T2\tabularnewline
TimeoutEventAck & smallint unsigned & 事件响应超时 & CSA 6.2
T3\tabularnewline
TimeoutIdle & smallint unsigned & 通信空闲时间 & CSA 6.3
T4\tabularnewline
TimeoutKeepaliveAck & smallint unsigned & 心跳响应超时 & CSA 6.3
T5\tabularnewline
RetryCmd & tinyint unsigned & 指令重试次数 & CSA 6.1 N1\tabularnewline
RetryEvent & tinyint unsigned & 事件重试次数 & CSA 6.2 N2\tabularnewline
RetryKeepalive & tinyint unsigned & 心跳重试次数 & CSA 6.3
N3\tabularnewline
MapFormat & int & 地图类型 & 0: BMP, 1: GIS\tabularnewline
EmergencyMode & int & 紧急调光模式 & 0: default brightness, 1: last
brightness, 2: schedule\tabularnewline
LastDeviceModeChange & timestamp & DeviceMode 表最后更新时间
&\tabularnewline
LastAutoPolicyChange & timestamp & AutoPolicy 表最后更新时间
&\tabularnewline
LastTimeScheduleChange & timestamp & TimeSchedule 表最后更新时间
&\tabularnewline
LastFrontendSceneChange & timestamp & FrontendScene 表最后更新时间
&\tabularnewline
\bottomrule
\end{longtabu}

\subsection{User}\label{user}

\begin{longtabu}[c]{@{}lXXX@{}}
\toprule
字段名 & 类型 & 含义 & 备注\tabularnewline
\midrule
\endhead
idUser & int unsigned & &\tabularnewline
Username & varchar(64) & &\tabularnewline
MD5 & varchar(40) & MD5(username+password) & 128-bit, 32 digits long hex
string\tabularnewline
Admin & tinyint & 用户权限 & 0: StarRiver服务; 1: 一般用户, 2:
管理员\tabularnewline
Note & varchar(450) & 备注 & UTF8\tabularnewline
\bottomrule
\end{longtabu}

默认分配如下两个用户:

\begin{longtabu}[c]{@{}lll@{}}
\toprule
idUser & Username & 备注\tabularnewline
\midrule
\endhead
0 & comm & StarRiver服务\tabularnewline
1 & admin & 管理员\tabularnewline
\bottomrule
\end{longtabu}

\subsection{Controller}\label{controller}

\begin{longtabu}[c]{@{}lXXX@{}}
\toprule
字段名 & 类型 & 含义 & 备注\tabularnewline
\midrule
\endhead
idController & int unsigned & &\tabularnewline
Name & varchar(128) & &\tabularnewline
MAC & binary(8) & 8字节控制器地址 &\tabularnewline
CommProtocol & tinyint unsigned & 通信协议 & 0: Shanghai, 1: Sansi
LC300, 2: CSA\tabularnewline
DataProtocol & tinyint unsigned & 通信方式 & 0: 串口, 1: TCP, 2:
UDP\tabularnewline
ComPort & varchar(8) & 串口号 & COM0, COM1, ttyS0,
ttyS1等\tabularnewline
ComBaud & int unsigned & 串口波特率 & 9600, 115200等\tabularnewline
ComAddr & tinyint unsigned & 使用485串口通信时指明地址 &\tabularnewline
IPAddr & varchar(16) & & xxx.xxx.xxx.xxx\tabularnewline
IPPort & smallint unsigned & &\tabularnewline
Note & varchar(450) & 备注 & 比如位置信息\tabularnewline
DisplayOrder & int unsigned & 前台程序界面上显示的顺序 &
不宜重复\tabularnewline
\bottomrule
\end{longtabu}

\subsection{ControllerStatus}\label{controllerstatus}

状态量不支持查询的话对应字段写 \texttt{NULL}。

\begin{longtabu}[c]{@{}lXXX@{}}
\toprule
字段名 & 类型 & 含义 & 备注\tabularnewline
\midrule
\endhead
idController & int unsigned & Controller.idController &\tabularnewline
CommState & tinyint unsigned & 通信状态 & 0: 正常, 1:
故障\tabularnewline
LoginState & tinyint unsigned & 是否已注册到StarRiver服务服务端 & 0:
未注册, 1: 已注册\tabularnewline
VersionSoftware & binary(4) & 软件版本 &\tabularnewline
VersionSystem & binary(4) & 系统版本 &\tabularnewline
VersionKernel & binary(4) & 内核版本 &\tabularnewline
VersionHardware & binary(4) & 硬件版本 &\tabularnewline
ManufacturerInfo & varchar(450) & 厂商信息文本内容 & UTF8\tabularnewline
Type & binary(2) & 控制器类型 & 上海:0x0001 =\textgreater{} 485控制器,
0x0002 =\textgreater{} PLC控制器, 0x0003 =\textgreater{}
无线控制器。LC300 同上\tabularnewline
WorkMode & binary(1) & 工作模式 & 上海:0x01 =\textgreater{}
远程控制模式, 0x02 =\textgreater{}
时控工作模式。LC300同上。SrConfig程序初始化数据库时写0xff,代表控制器尚未查询过。\tabularnewline
StatusCode & binary(1) & 控制器状态代码 & 上海:0x00 =\textgreater{}
正常工作, 0x01 =\textgreater{} 下行通道异常, 0x02 =\textgreater{}
升级模式, 0xFF =\textgreater{} 未知错误。LC300未描述。\tabularnewline
LastTime & datetime & 最后一次从控制器读取的时间 &
用于检查控制器时间是否正常\tabularnewline
SmsCount & smallint unsigned & 短信数量 &
从控制器读回来的\tabularnewline
\bottomrule
\end{longtabu}

\subsection{ControllerStatusChanges}\label{controllerstatuschanges}

记录控制器状态变化。

\begin{longtabu}[c]{@{}lXXX@{}}
\toprule
字段名 & 类型 & 含义 & 备注\tabularnewline
\midrule
\endhead
idController & int unsigned & Controller.idController &
PK1\tabularnewline
field & varchar(32) & 被修改的字段名 & PK2\tabularnewline
field\_value & varchar(255) & 变化后的字段值\tabularnewline
time & timestamp & 修改时间 & PK3\tabularnewline
\bottomrule
\end{longtabu}

目前记录如下两个状态的变化,由 \texttt{ControllerStatus} 表中名为
\texttt{log\_controller\_changes} 的trigger执行:

\begin{enumerate}
\def\labelenumi{\arabic{enumi}.}
\itemsep1pt\parskip0pt\parsep0pt
\item
  CommState: \texttt{field\_value} 存储 \texttt{CAST(CommState AS char)}
\item
  StatusCode: \texttt{field\_value} 存储 \texttt{HEX(StatusCode)}
\end{enumerate}

每月执行的事件 \texttt{Delete outdated status history}
会将超过365天前的记录删除。

\subsection{ControllerSchedule}\label{controllerschedule}

查询时控计划时把结果放到这张表里,将原来该控制器的删掉,再重新插入。

\begin{longtabu}[c]{@{}lXXX@{}}
\toprule
字段名 & 类型 & 含义 & 备注\tabularnewline
\midrule
\endhead
id & bigint unsigned & 自增ID &\tabularnewline
idController & int unsigned & Controller.idController &\tabularnewline
Item & varchar(128) & 时间亮度表的一项 & 上海:MM, DD, MM, DD, HH, mm,
ss, Addr, mode, value (逗号后不加空格)\tabularnewline
\bottomrule
\end{longtabu}

\subsection{ControllerSchedule\_edit}\label{controllerscheduleux5fedit}

用户编辑时控计划时把结果放到这张表里。

\begin{longtabu}[c]{@{}lXXX@{}}
\toprule
字段名 & 类型 & 含义 & 备注\tabularnewline
\midrule
\endhead
id & bigint unsigned & 自增ID &\tabularnewline
idController & int unsigned & Controller.idController &\tabularnewline
Item & varchar(128) & 时间亮度表的一项 & 上海:MM, DD, MM, DD, HH, mm,
ss, Addr, mode, value (逗号后不加空格)\tabularnewline
\bottomrule
\end{longtabu}

\subsection{ControllerCellNumbers}\label{controllercellnumbers}

控制器上的手机号。
往控制器上面上载手机号,或者从控制器上下载手机号,都是一帧就完成了。
删除手机号是把所有的手机号删除,不能只删除其中的一个。

\begin{longtabu}[c]{@{}lXXX@{}}
\toprule
字段名 & 类型 & 含义 & 备注\tabularnewline
\midrule
\endhead
id & bigint unsigned & 自增ID &\tabularnewline
idController & int unsigned & Controller.idController &\tabularnewline
Cell & varchar(20) & 一个手机号 & 不支持国家代码\tabularnewline
\bottomrule
\end{longtabu}

\subsection{ControllerSms}\label{controllersms}

从控制器里面读回来的短信内容放在这张表里面。

\begin{longtabu}[c]{@{}lXXX@{}}
\toprule
字段名 & 类型 & 含义 & 备注\tabularnewline
\midrule
\endhead
idController & int unsigned & Controller.idController &
PK1\tabularnewline
SmsIndex & smallint unsigned & 短信在设备中的序号 & PK2\tabularnewline
SmsContent & varchar(450) & & UTF8\tabularnewline
\bottomrule
\end{longtabu}

\subsection{ControllerLog}\label{controllerlog}

从控制器里面读回来的日志内容放在这张表里面。

\begin{longtabu}[c]{@{}lXXX@{}}
\toprule
字段名 & 类型 & 含义 & 备注\tabularnewline
\midrule
\endhead
idController & int unsigned & Controller.idController &
PK1\tabularnewline
LogTime & datetime & 日志记录时间 & PK2\tabularnewline
LogContent & text & 日志内容 & UTF8\tabularnewline
\bottomrule
\end{longtabu}

\subsection{Device}\label{device}

\begin{longtabu}[c]{@{}lXXX@{}}
\toprule
字段名 & 类型 & 含义 & 备注\tabularnewline
\midrule
\endhead
idDevice & int unsigned & &\tabularnewline
Name & varchar(128) & &\tabularnewline
MAC & binary(8) & &\tabularnewline
DisplayOrder & int unsigned & 前台程序界面上显示的顺序 &\tabularnewline
Note & varchar(450) & 备注,比如位置信息 & UTF8\tabularnewline
\bottomrule
\end{longtabu}

\subsection{DeviceStatus}\label{devicestatus}

\begin{longtabu}[c]{@{}lXXX@{}}
\toprule
字段名 & 类型 & 含义 & 备注\tabularnewline
\midrule
\endhead
idDevice & int unsigned & Device.idDevice &\tabularnewline
VersionSoftware & binary(4) & 软件版本号 &
LC300:只用2字节\tabularnewline
VersionHardware & binary(4) & 硬件版本号 &
LC300:只用2字节\tabularnewline
ManufacturerInfo & varchar(450) & 厂商信息文本 &
上海:厂商描述。LC300:终端设备产品信息,\textless{}=32字节,ASCII码。\tabularnewline
ManufactureTime & datetime & 出厂时间 & LC300:无\tabularnewline
Type & binary(4) & 设备类型 & 上海4.4.1 LC300 4.6.1
\texttt{0x0E}\tabularnewline
SN & varchar(64) & 产品序列号 &\tabularnewline
InputVolt & double & 输入电压采样值 &\tabularnewline
InputAmp & double & 输入电流采样值 &\tabularnewline
OutputVolt & double & 输出电压采样值 &\tabularnewline
OutputAmp & double & 输出电流采样值 &\tabularnewline
ActivePower & double & 有功功率采样值 &\tabularnewline
Temperature & smallint & 温度采样值 & LC300:1字节\tabularnewline
TemperatureGuard & binary(4) & 过温保护状态 过温保护参数 & LC300 4.6
\texttt{0x0A}\tabularnewline
Uptime & int unsigned & 上电工作时间 &\tabularnewline
UptimeTotal & int unsigned & 总工作时间 &\tabularnewline
ElectricityConsumption & int unsigned & 消耗电量值 &\tabularnewline
TransitionTime & tinyint unsigned & 调光渐变时间 &\tabularnewline
Brightness & tinyint unsigned & 当前亮度 &\tabularnewline
BrightnessMin & tinyint unsigned & 最小亮度值 &\tabularnewline
BrightnessMax & tinyint unsigned & 最大亮度值 &\tabularnewline
BrightnessMinPhysical & tinyint unsigned & 物理最小亮度值
&\tabularnewline
BrightnessMaxPhysical & tinyint unsigned & 物理最大亮度值 &
LC300未定义\tabularnewline
BrightnessPowerOn & tinyint unsigned & 上电亮度值 &\tabularnewline
BrightnessDefault & tinyint unsigned & 默认故障亮度值 &\tabularnewline
BrightnessCoefficiency & tinyint unsigned & 调光系数 & 实际亮度 =
设置亮度 * 调光系数 / 100\tabularnewline
Status & binary(1) & 设备状态 & 上海:4.5.1 LC300:4.6
\texttt{0x06}\tabularnewline
StatusComm & binary(1) & 通讯状态 & 上海:4.5.1\tabularnewline
StatusLamp & binary(1) & 灯具状态 & 上海:4.5.1\tabularnewline
SensorI & binary(4) & 光强传感器采样值 &
上海:光感亮度采样值\tabularnewline
LC300:终端传感器设备光强值\tabularnewline
SensorL & binary(4) & 光照传感器采样值 &
LC300:终端传感器设备光照值\tabularnewline
SensorH & binary(4) & 湿度传感器采样值 &\tabularnewline
SensorT & binary(4) & 车流量传感器采样值 &\tabularnewline
GroupMask & binary(32) & 分组掩码 & 一共256
bit,若灯属于第N个组,则bit(N)=1。上海:组地址是1-254。LC300:组地址是0-63\tabularnewline
\bottomrule
\end{longtabu}

\subsection{DeviceStatus\_edit}\label{devicestatusux5fedit}

StarRiver Config 在此记录用户录入的设备初始信息。

\begin{longtabu}[c]{@{}lXXX@{}}
\toprule
字段名 & 类型 & 含义 & 备注\tabularnewline
\midrule
\endhead
idDevice & int unsigned & Device.idDevice &\tabularnewline
TransitionTime & tinyint unsigned & 调光渐变时间 &\tabularnewline
BrightnessMin & tinyint unsigned & 最小亮度值 &\tabularnewline
BrightnessMax & tinyint unsigned & 最大亮度值 &\tabularnewline
BrightnessMinPhysical & tinyint unsigned & 物理最小亮度值
&\tabularnewline
BrightnessMaxPhysical & tinyint unsigned & 物理最大亮度值 &
LC300未定义\tabularnewline
BrightnessPowerOn & tinyint unsigned & 上电亮度值 &\tabularnewline
BrightnessDefault & tinyint unsigned & 默认故障亮度值 &\tabularnewline
BrightnessCoefficiency & tinyint unsigned & 调光系数 & 实际亮度 =
设置亮度 * 调光系数 / 100\tabularnewline
GroupMask & binary(32) & 分组掩码 & 一共256
bit,若灯属于第N个组,则bit(N)=1。上海:组地址是1-254。LC300:组地址是0-63。\tabularnewline
\bottomrule
\end{longtabu}

\subsection{DeviceStatusChanges}\label{devicestatuschanges}

记录设备状态变化。

\begin{longtabu}[c]{@{}lXXX@{}}
\toprule
字段名 & 类型 & 含义 & 备注\tabularnewline
\midrule
\endhead
idDevice & int unsigned & Device.idDevice & PK1\tabularnewline
field & varchar(32) & 被修改的字段名 & PK2\tabularnewline
field\_value & varchar(255) & 变化后的字段值 &\tabularnewline
time & timestamp & 修改时间 & PK3\tabularnewline
\bottomrule
\end{longtabu}

目前记录如下三个状态的变化,由DeviceStatus表中名为log\_changes的trigger执行:

\begin{enumerate}
\def\labelenumi{\arabic{enumi}.}
\itemsep1pt\parskip0pt\parsep0pt
\item
  Status: \texttt{field\_value} 存储 \texttt{HEX(Status)}
\item
  StatusComm: \texttt{field\_value} 存储 \texttt{HEX(StatusComm)}
\item
  StatusLamp: \texttt{field\_value} 存储 \texttt{HEX(StatusLamp)}
\end{enumerate}

\subsection{DeviceStatusHistory*}\label{devicestatushistory}

以下三张表的字段定义大部分相同,分别记录不同粒度的状态量历史。

\begin{enumerate}
\def\labelenumi{\arabic{enumi}.}
\itemsep1pt\parskip0pt\parsep0pt
\item
  DeviceStatusHistory
\item
  DeviceStatusHistory\_hourly
\item
  DeviceStatusHistory\_daily
\end{enumerate}

三张表共同部分的定义如下:

\begin{longtabu}[c]{@{}lXXX@{}}
\toprule
字段名 & 类型 & 含义 & 备注\tabularnewline
\midrule
\endhead
idDevice & int unsigned & Device.idDevice & PK1\tabularnewline
field & varchar(32) & 被修改的字段名 & PK2\tabularnewline
field\_value & double & 字段采样值 &\tabularnewline
time & timestamp & 修改时间 & PK3\tabularnewline
\bottomrule
\end{longtabu}

两张汇总表有额外记录汇总时段内峰值、谷值的字段。

\begin{enumerate}
\def\labelenumi{\arabic{enumi}.}
\itemsep1pt\parskip0pt\parsep0pt
\item
  DeviceStatusHistory\_hourly
\item
  DeviceStatusHistory\_daily
\end{enumerate}

峰谷值定义如下:

\begin{longtabu}[c]{@{}lXXX@{}}
\toprule
字段名 & 类型 & 含义 & 备注\tabularnewline
\midrule
\endhead
field\_max & double & 汇总时段内最大采样值 &
浮点数可能无法精确表示整数采样\tabularnewline
field\_min & double & 汇总时段内最小采样值 &
浮点数可能无法精确表示整数采样\tabularnewline
\bottomrule
\end{longtabu}

有三个event更新这些表:

\begin{enumerate}
\def\labelenumi{\arabic{enumi}.}
\itemsep1pt\parskip0pt\parsep0pt
\item
  每5分钟,对 \texttt{DeviceStatus} 进行采样,记录设备的状态量。
\item
  每天,将前一天的采样归纳成每小时均值。删除3天前的采样。
\item
  每周,将前一周的采样归纳成每日均值。删除一周前的小时均值。
\end{enumerate}

每月执行的事件 \texttt{Delete outdated status history} 会将
\texttt{DeviceStatusHistory\_daily} 中超过365天前的记录删除。

\subsection{DeviceSchedule}\label{deviceschedule}

LCP-SH-D:该功能是在单灯控制器与集中控制器失去联系后采用的异常模式,如果在某一时间段内没有设置该亮度值的话,将采用默认故障亮度值显示。

\begin{quote}
\textbf{注意} 调光计划的总数不能超过7。
\end{quote}

LC300没有定义。

\begin{longtabu}[c]{@{}lXXX@{}}
\toprule
字段名 & 类型 & 含义 & 备注\tabularnewline
\midrule
\endhead
idDeviceSchedule & bigint unsigned & 自增ID &\tabularnewline
idDevice & int unsigned & Device.idDevice &\tabularnewline
Item & varchar(32) & 自动亮度表的一项 & HH,mm,value\tabularnewline
\bottomrule
\end{longtabu}

\subsection{DeviceScene}\label{devicescene}

LCP-SH-D协议场景号是1-254。 LC300协议,场景号是1-16,0表示无场景。

\begin{longtabu}[c]{@{}lXXX@{}}
\toprule
字段名 & 类型 & 含义 & 备注\tabularnewline
\midrule
\endhead
idDeviceScene & tinyint unsigned & &\tabularnewline
idDevice & int unsigned & Device.idDevice &\tabularnewline
Brightness & tinyint unsigned & 灯的场景亮度值 & 0-100\tabularnewline
\bottomrule
\end{longtabu}

\subsection{DeviceReportCond}\label{devicereportcond}

信息上报条件。用于保存要给哪些灯设置信息上报条件。

LC300未定义。

\begin{longtabu}[c]{@{}lXXX@{}}
\toprule
字段名 & 类型 & 含义 & 备注\tabularnewline
\midrule
\endhead
id & bigint unsigned & 自增ID &\tabularnewline
idDevice & int unsigned & Device.idDevice &\tabularnewline
Item & varchar(128) & 信息上报的一项 & StatusID,low,high\tabularnewline
\bottomrule
\end{longtabu}

\subsection{DeviceAlarmCond}\label{devicealarmcond}

报警阈值条件。用于保存要给哪些灯设置报警阈值条件。

LC300未定义。

\begin{longtabu}[c]{@{}lXXX@{}}
\toprule
字段名 & 类型 & 含义 & 备注\tabularnewline
\midrule
\endhead
id & bigint unsigned & 自增ID &\tabularnewline
idDevice & int unsigned & Device.idDevice &\tabularnewline
Item & varchar(450) & 报警阈值的一项 & StatusID, low, high, warn\_msg
(逗号后不加空格)\tabularnewline
\bottomrule
\end{longtabu}

\subsection{DeviceMessages}\label{devicemessages}

和上面两张表相关,上面设置了信息上报条件,这样对上报上来的信息应该有一个地方记住。同一个类型的上报报警信息,如果报了两条,则后面的那条应该把前面那条覆盖掉。

LC300未定义。

\begin{longtabu}[c]{@{}lXXX@{}}
\toprule
字段名 & 类型 & 含义 & 备注\tabularnewline
\midrule
\endhead
id & bigint unsigned & 自增ID &\tabularnewline
idDevice & int unsigned & Device.idDevice &\tabularnewline
MsgTime & timestamp & &\tabularnewline
Item & varchar(450) & 报警阈值的一项 & StatusID,value,msg
(若非报警,msg留空)\tabularnewline
\bottomrule
\end{longtabu}

\subsection{TaskTodo}\label{tasktodo}

待处理的用户操作。

服务器进程每隔一段时间在这里取任务,分别处理。

\begin{longtabu}[c]{@{}lXXX@{}}
\toprule
字段名 & 类型 & 含义 & 备注\tabularnewline
\midrule
\endhead
id & bigint unsigned & 自增ID & 当流水号\tabularnewline
Time & datetime & &\tabularnewline
idUser & int unsigned & User.idUser &\tabularnewline
Command & smallint unsigned & 操作类型 &\tabularnewline
Parameters & varchar(1024) & 操作的参数 &\tabularnewline
MD5 & varchar(40) & 命令的MD5散列值 & md5(Time + idUser + Command +
Parameters)
作为查询条件供前段程序在TaskDone中查询任务完成状态\tabularnewline
\bottomrule
\end{longtabu}

\begin{quote}
\textbf{注意}:这个表使用MyISAM引擎存储。自增的id字段如果使用InnoDB存储,计数器在内存里,数据库服务器一旦重启,又将从1开始计数,完成任务后向
\texttt{TaskDone} 中插入记录会引发冲突导致失败。
\end{quote}

\subsection{TaskDone}\label{taskdone}

记录那些已完成的用户操作。

\texttt{TaskTodo} 中的任务,处理完毕后,从 \texttt{TaskTodo}
删除,将结果插入到表 \texttt{TaskDone} ,保持 \texttt{TaskTodo}
中字段信息不变。

除了 \texttt{TaskTodo} 的字段,另有以下字段:

\begin{longtabu}[c]{@{}lXXX@{}}
\toprule
字段名 & 类型 & 含义 & 备注\tabularnewline
\midrule
\endhead
FinishedTime & timestamp & 完成时间 &
插入记录时由数据库自动生成\tabularnewline
ReturnCode & smallint & 返回代码 & 成功:0;失败:控制器返回的确认码
bitwise-OR;控制器未能返回结果(如超时):-2。\tabularnewline
ErrorMsg & varchar(255) & 错误描述信息 & UTF8\tabularnewline
\bottomrule
\end{longtabu}

有一个 event (\emph{Remove outdated tasks from TaskDone})
每周清理创建时间在14天之前的任务,以保证这张表不会无限膨胀下去。

\subsection{Firmware}\label{firmware}

\begin{longtabu}[c]{@{}lXXX@{}}
\toprule
字段名 & 类型 & 含义 & 备注\tabularnewline
\midrule
\endhead
idFirmware & int unsigned & &\tabularnewline
Firmware & mediumblob & 二进制内容 &\tabularnewline
FirmwareMD5 & char(32) & md5(Firmware) &\tabularnewline
FirmwareName & varchar(64) & 固件包名称 & 终端用\tabularnewline
FirmwareVersion & binary(4) & 固件版本号 & 终端用\tabularnewline
Note & varchar(300) & & UTF8\tabularnewline
\bottomrule
\end{longtabu}

MediumBlob可以存储16MB大小的固件,MySQL服务器也已经设置:

\texttt{max\_allowed\_packet=16M}

\subsection{FrontendMapBmp}\label{frontendmapbmp}

\begin{longtabu}[c]{@{}lXXX@{}}
\toprule
字段名 & 类型 & 含义 & 备注\tabularnewline
\midrule
\endhead
idFrontendMapBmp & tinyint unsigned & &\tabularnewline
Name & varchar(64) & &\tabularnewline
DisplayOrder & tinyint unsigned & 前台程序界面上显示地图的顺序 &
不宜重复\tabularnewline
\bottomrule
\end{longtabu}

\subsection{FrontendGroup}\label{frontendgroup}

对于LCP-SH-D和LC300协议:现在用户的组和灯的组是统一的,即用户的第一个组就是所有灯的第一个组;对于CSA来说,一个用户的组可能对应控制器1的网关ID1和控制器2的网关ID2,所以还要建一张表来指定用户组和控制器里网关ID的关系(有点像以前LC200)。

\begin{longtabu}[c]{@{}lXXX@{}}
\toprule
字段名 & 类型 & 含义 & 备注\tabularnewline
\midrule
\endhead
idFrontendGroup & tinyint unsigned & &\tabularnewline
Name & varchar(64) & &\tabularnewline
Color & tinyint unsigned & & Config程序里作为选项\tabularnewline
\bottomrule
\end{longtabu}

\subsection{FrontendScene}\label{frontendscene}

这是用户定义的场景。

\begin{longtabu}[c]{@{}lXXX@{}}
\toprule
字段名 & 类型 & 含义 & 备注\tabularnewline
\midrule
\endhead
idFrontendScene & int unsigned & &\tabularnewline
Name & varchar(64) & &\tabularnewline
DisplayOrder & int unsigned & & 不宜重复\tabularnewline
\bottomrule
\end{longtabu}

\subsection{FrontendSceneItem}\label{frontendsceneitem}

这是用户定义的场景。

\begin{longtabu}[c]{@{}lXXX@{}}
\toprule
字段名 & 类型 & 含义 & 备注\tabularnewline
\midrule
\endhead
id & bigint unsigned & &\tabularnewline
scene\_id & int unsigned & &\tabularnewline
device\_addr & binary(8) & &
可以是单灯地址,也可以是组地址\tabularnewline
mode & int unsigned & & 0: manual, 1: auto, 2: schedule\tabularnewline
value & int unsigned & & brightness OR id auto policy OR id
schedule\tabularnewline
\bottomrule
\end{longtabu}

\subsection{FrontendControllerDevices}\label{frontendcontrollerdevices}

描述灯属于哪个控制器。

对于CSA的协议,灯是属于控制器下的某个网关ID,所以可能会增加一个字段表示灯属于控制器的哪个网关。

\begin{longtabu}[c]{@{}lXXX@{}}
\toprule
字段名 & 类型 & 含义 & 备注\tabularnewline
\midrule
\endhead
idController & int unsigned & Controller.idController &
PK1\tabularnewline
idDevice & int unsigned & Device.idDevice & PK2\tabularnewline
DeviceDisplayOrder & int unsigned & 界面显示该控制器下的灯时的序号
&\tabularnewline
PortOnController & smallint unsigned & 灯在控制器的哪个口上 &
可能不需要\tabularnewline
\bottomrule
\end{longtabu}

\subsection{FrontendGroupDevices}\label{frontendgroupdevices}

描述灯属于哪个组。

\begin{longtabu}[c]{@{}lXXX@{}}
\toprule
字段名 & 类型 & 含义 & 备注\tabularnewline
\midrule
\endhead
idFrontendGroup & tinyint unsigned & FrontendGroup. idFrontendGroup &
PK1\tabularnewline
idDevice & int unsigned & Device.idDevice & PK2\tabularnewline
DeviceDisplayOrder & int unsigned & 界面显示该组内的灯时的序号 &
不宜重复\tabularnewline
\bottomrule
\end{longtabu}

\subsection{FrontendDeviceMap}\label{frontenddevicemap}

描述各种设备在BMP地图上的信息。

\begin{longtabu}[c]{@{}lXXX@{}}
\toprule
字段名 & 类型 & 含义 & 备注\tabularnewline
\midrule
\endhead
idFrontendMapBmp & int unsigned & FrontendMapBmp. idFrontendMapBmp &
PK1\tabularnewline
DeviceType & tinyint unsigned & 表明设备类型,如控制器、灯等 &
PK2\tabularnewline
idDevice & int unsigned & Device.idDevice or Controller.idController &
PK3\tabularnewline
PosX & int & 设备在地图上的横坐标 &\tabularnewline
PosY & int & 设备在地图上的纵坐标 &\tabularnewline
idIcon & tinyint unsigned & & 对应的图标文件必须存在\tabularnewline
\bottomrule
\end{longtabu}

\subsection{Sensor}\label{sensor}

传感器信息。

\begin{longtabu}[c]{@{}lXXX@{}}
\toprule
字段名 & 类型 & 含义 & 备注\tabularnewline
\midrule
\endhead
id & int unsigned & &\tabularnewline
name & varchar(255) & &\tabularnewline
type & int unsigned & & 0: virtual, 1: brightness, 2:
traffic\tabularnewline
source & varchar(255) & & ``controller\_mac,port(1-4)'' for those
attached on a controller; ``ip,port,user,pass,sql'' for those available
in a database\tabularnewline
raw\_value & double & &\tabularnewline
normalized\_value & int & & mapped to range {[}0,100{]}\tabularnewline
normalize\_method & int & & predefined method id here. Method lies in
source code.\tabularnewline
\bottomrule
\end{longtabu}

\subsection{AutoPolicy}\label{autopolicy}

自动调光策略。

\begin{longtabu}[c]{@{}lXXX@{}}
\toprule
字段名 & 类型 & 含义 & 备注\tabularnewline
\midrule
\endhead
id & int unsigned & &\tabularnewline
name & varchar(255) & &\tabularnewline
display\_order & int unsigned & &\tabularnewline
sensor\_id & int unsigned & Sensor.id &\tabularnewline
\bottomrule
\end{longtabu}

\subsection{AutoPolicyItem}\label{autopolicyitem}

自动调光策略单项。

\begin{longtabu}[c]{@{}lXXX@{}}
\toprule
字段名 & 类型 & 含义 & 备注\tabularnewline
\midrule
\endhead
id & bigint unsigned & &\tabularnewline
id\_policy & int unsigned & AutoPolicy.id &\tabularnewline
input & bigint unsigned & & {[}0,100{]}\tabularnewline
output & bigint unsigned & & {[}0,100{]}\tabularnewline
type & bigint unsigned & & 0: calculated, 1: user defined\tabularnewline
\bottomrule
\end{longtabu}

对于每个 \texttt{id\_policy} ,都应生成101个策略项,分别对应
\texttt{input}
为0到100的情形。其中若干项是用户设定的,其余是插值计算而得。

\subsection{TimeSchedule}\label{timeschedule}

调光时间表。

\begin{longtabu}[c]{@{}lXXX@{}}
\toprule
字段名 & 类型 & 含义 & 备注\tabularnewline
\midrule
\endhead
id & int unsigned & &\tabularnewline
name & varchar(255) & &\tabularnewline
display\_order & int unsigned & &\tabularnewline
\bottomrule
\end{longtabu}

\subsection{TimeScheduleItem}\label{timescheduleitem}

调光时间表的逐条记录。

\begin{longtabu}[c]{@{}lXXX@{}}
\toprule
字段名 & 类型 & 含义 & 备注\tabularnewline
\midrule
\endhead
id & bigint unsigned & &\tabularnewline
idTimeSchedule & int unsigned & &\tabularnewline
name & varchar(255) & & MM, DD, MM, DD, HH:mm:ss,
brightness\tabularnewline
\bottomrule
\end{longtabu}

\subsection{DeviceMode}\label{devicemode}

各控制器下各个设备的调光模式

\begin{longtabu}[c]{@{}lXXX@{}}
\toprule
字段名 & 类型 & 含义 & 备注\tabularnewline
\midrule
\endhead
controller\_addr & binary(8) & & PK1\tabularnewline
device\_addr & binary(8) & & PK2\tabularnewline
mode & int unsigned & & 0: manual, 1: auto, 2: schedule\tabularnewline
\bottomrule
\end{longtabu}
