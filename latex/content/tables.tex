\section{表结构}\label{ux8868ux7ed3ux6784}

除特殊说明外,每张表的第一个字段为主键 (Primary Key)。

\subsection{system\_config}\label{systemux5fconfig}

\begin{longtabu}[c]{@{}lXXX@{}}
\toprule
字段名 & 类型 & 含义 & 备注\tabularnewline
\midrule
\endhead
id & int unsigned & & AI PK\tabularnewline
network\_mode & tinyint unsigned & StarRiver服务角色 & 0: 服务器, 1:
客户端。考虑同时作为多种角色与不同控制器通信的需求\tabularnewline
listening\_port & smallint unsigned & 服务器侦听端口 &\tabularnewline
last\_server\_activity & timest\_amp & StarRiver服务最近连接时间 &
StarRiver服务定时更新这个字段。\tabularnewline
interval\_report\_activity & int unsigned & 更新最近连接时间间隔
&\tabularnewline
interval\_status\_query & int unsigned & 自动查询控制器状态间隔
&\tabularnewline
interval\_device\_status\_query & int unsigned & 自动查询设备状态间隔
&\tabularnewline
interval\_retry\_write & int unsigned & 连接建立期间,写操作重试间隔 &
单位为毫秒\tabularnewline
interval\_keepalive & int unsigned & 心跳包发送间隔 &\tabularnewline
timeout\_cmd\_ack & smallint unsigned & 指令响应超时 & CSA 6.1
T1\tabularnewline
timeout\_cmd\_result & smallint unsigned & 指令结果超时 & CSA 6.1
T2\tabularnewline
timeout\_event\_ack & smallint unsigned & 事件响应超时 & CSA 6.2
T3\tabularnewline
timeout\_idle & smallint unsigned & 通信空闲时间 & CSA 6.3
T4\tabularnewline
timeout\_keepalive\_ack & smallint unsigned & 心跳响应超时 & CSA 6.3
T5\tabularnewline
cmd\_retries & tinyint unsigned & 指令重试次数 & CSA 6.1
N1\tabularnewline
event\_retries & tinyint unsigned & 事件重试次数 & CSA 6.2
N2\tabularnewline
keepalive\_retries & tinyint unsigned & 心跳重试次数 & CSA 6.3
N3\tabularnewline
map\_format & int & 地图类型 & 0: BMP, 1: GIS\tabularnewline
emergency\_mode & int & 紧急调光模式 & 0: default brightness, 1: last
brightness, 2: schedule\tabularnewline
ts\_device\_mode & timest\_amp & device\_mode 表最后更新时间 &
由服务端更新\tabularnewline
ts\_auto\_policy & timest\_amp & auto\_policy 表最后更新时间 &
由客户端更新\tabularnewline
ts\_time\_schedule & timest\_amp & time\_schedule 表最后更新时间 &
由客户端更新\tabularnewline
ts\_frontend\_scene & timest\_amp & frontend\_scene 表最后更新时间 &
由客户端更新\tabularnewline
\bottomrule
\end{longtabu}

\subsection{user}\label{user}

\begin{longtabu}[c]{@{}lXXX@{}}
\toprule
字段名 & 类型 & 含义 & 备注\tabularnewline
\midrule
\endhead
id & int unsigned & &\tabularnewline
username & varchar(64) & &\tabularnewline
md5 & varchar(40) & MD5(username+password) & 128-bit, 32 digits long hex
string\tabularnewline
permission & tinyint & 用户权限 & 0: StarRiver服务; 1: 一般用户, 2:
管理员\tabularnewline
first\_name & varchar(64) & &\tabularnewline
last\_name & varchar(64) & &\tabularnewline
email & varchar(128) & &\tabularnewline
registration\_key & varchar(128) & & 预留注册用户名时使用\tabularnewline
reset\_password\_key & varchar(128) & &
预留重置密码时使用\tabularnewline
registration\_id & varchar(128) & & 预留注册用户名时使用\tabularnewline
note & varchar(450) & 备注 & UTF8\tabularnewline
\bottomrule
\end{longtabu}

默认分配如下两个用户:

\begin{longtabu}[c]{@{}lll@{}}
\toprule
id & username & note\tabularnewline
\midrule
\endhead
0 & comm & StarRiver服务\tabularnewline
1 & admin & 管理员\tabularnewline
\bottomrule
\end{longtabu}

\subsection{controller}\label{controller}

\begin{longtabu}[c]{@{}lXXX@{}}
\toprule
字段名 & 类型 & 含义 & 备注\tabularnewline
\midrule
\endhead
id & int unsigned & &\tabularnewline
name & varchar(128) & &\tabularnewline
addr & binary(8) & 8字节控制器地址 &\tabularnewline
comm\_protocol & tinyint unsigned & 通信协议 & 0: Shanghai, 1: Sansi
LC300, 2: CSA\tabularnewline
data\_protocol & tinyint unsigned & 通信方式 & 0: 串口, 1: TCP, 2:
UDP\tabularnewline
com\_port & varchar(8) & 串口号 & COM0, COM1, ttyS0,
ttyS1等\tabularnewline
com\_baud & int unsigned & 串口波特率 & 9600, 115200等\tabularnewline
com\_addr & tinyint unsigned & 使用485串口通信时指明地址
&\tabularnewline
ip\_addr & varchar(16) & & xxx.xxx.xxx.xxx\tabularnewline
ip\_port & smallint unsigned & &\tabularnewline
note & varchar(450) & 备注 & 比如位置信息\tabularnewline
display\_order & int unsigned & 前台程序界面上显示的顺序 &
不宜重复\tabularnewline
\bottomrule
\end{longtabu}

\subsection{controller\_status}\label{controllerux5fstatus}

状态量不支持查询的话对应字段写 \texttt{NULL}。

\begin{longtabu}[c]{@{}lXXX@{}}
\toprule
字段名 & 类型 & 含义 & 备注\tabularnewline
\midrule
\endhead
controller\_id & int unsigned & controller.id &\tabularnewline
comm\_state & tinyint unsigned & 通信状态 & 0: 正常, 1:
故障\tabularnewline
login\_state & tinyint unsigned & 是否已注册到StarRiver服务服务端 & 0:
未注册, 1: 已注册\tabularnewline
version\_software & binary(4) & 软件版本 &\tabularnewline
version\_system & binary(4) & 系统版本 &\tabularnewline
version\_kernel & binary(4) & 内核版本 &\tabularnewline
version\_hardware & binary(4) & 硬件版本 &\tabularnewline
manufacturer\_info & varchar(450) & 厂商信息文本内容 &
UTF8\tabularnewline
type & binary(2) & 控制器类型 & 上海:\texttt{0x0001} =\textgreater{}
485控制器, \texttt{0x0002} =\textgreater{} PLC控制器, \texttt{0x0003}
=\textgreater{} 无线控制器。LC300 同上\tabularnewline
work\_mode & binary(1) & 工作模式 & 上海:\texttt{0x01} =\textgreater{}
远程控制模式, \texttt{0x02} =\textgreater{}
时控工作模式。LC300同上。SrConfig程序初始化数据库时写
\texttt{0xFF},代表控制器尚未查询过。\tabularnewline
status\_code & binary(1) & 控制器状态代码 & 上海:\texttt{0x00}
=\textgreater{} 正常工作, \texttt{0x01} =\textgreater{} 下行通道异常,
\texttt{0x02} =\textgreater{} 升级模式, \texttt{0xFF} =\textgreater{}
未知错误。LC300未描述。\tabularnewline
last\_time & datetime & 最后一次从控制器读取的时间 &
用于检查控制器时间是否正常\tabularnewline
sms\_count & smallint unsigned & 短信数量 &
从控制器读回来的\tabularnewline
\bottomrule
\end{longtabu}

\subsection{controller\_status\_changes}\label{controllerux5fstatusux5fchanges}

记录控制器状态变化。

\begin{longtabu}[c]{@{}lXXX@{}}
\toprule
字段名 & 类型 & 含义 & 备注\tabularnewline
\midrule
\endhead
controller\_id & int unsigned & controller.id & PK1\tabularnewline
field & varchar(32) & 被修改的字段名 & PK2\tabularnewline
field\_value & varchar(255) & 变化后的字段值 &\tabularnewline
time & timest\_amp & 修改时间 & PK3\tabularnewline
\bottomrule
\end{longtabu}

目前记录如下两个状态(亦即 \texttt{field} 的合法取值)的变化,由
\texttt{controller\_status} 表中名为 \texttt{log\_controller\_changes}
的trigger执行:

\begin{enumerate}
\def\labelenumi{\arabic{enumi}.}
\itemsep1pt\parskip0pt\parsep0pt
\item
  ``comm\_state'': \texttt{field\_value} 存储
  \texttt{CAST(comm\_state AS char)}
\item
  ``status\_code'': \texttt{field\_value} 存储
  \texttt{HEX(status\_code)}
\end{enumerate}

每月执行的事件 \texttt{Delete outdated status history}
会将超过365天前的记录删除。

\subsection{controller\_schedule}\label{controllerux5fschedule}

查询时控计划时把结果放到这张表里,将原来该控制器的删掉,再重新插入。

\begin{longtabu}[c]{@{}lXXX@{}}
\toprule
字段名 & 类型 & 含义 & 备注\tabularnewline
\midrule
\endhead
id & bigint unsigned & 自增ID &\tabularnewline
controller\_id & int unsigned & controller.id &\tabularnewline
item & varchar(128) & 时间亮度表的一项 & 上海:MM, DD, MM, DD, HH, mm,
ss, Addr, mode, value (逗号后不加空格)\tabularnewline
\bottomrule
\end{longtabu}

\subsection{controller\_schedule\_edit}\label{controllerux5fscheduleux5fedit}

用户编辑时控计划时把结果放到这张表里。

\begin{longtabu}[c]{@{}lXXX@{}}
\toprule
字段名 & 类型 & 含义 & 备注\tabularnewline
\midrule
\endhead
id & bigint unsigned & 自增ID &\tabularnewline
controller\_id & int unsigned & controller.id &\tabularnewline
item & varchar(128) & 时间亮度表的一项 & 上海:MM, DD, MM, DD, HH, mm,
ss, Addr, mode, value (逗号后不加空格)\tabularnewline
\bottomrule
\end{longtabu}

\subsection{device}\label{device}

\begin{longtabu}[c]{@{}lXXX@{}}
\toprule
字段名 & 类型 & 含义 & 备注\tabularnewline
\midrule
\endhead
id & int unsigned & &\tabularnewline
name & varchar(128) & &\tabularnewline
addr & binary(8) & &\tabularnewline
display\_order & int unsigned & 前台程序界面上显示的顺序
&\tabularnewline
note & varchar(450) & 备注,比如位置信息 & UTF8\tabularnewline
\bottomrule
\end{longtabu}

\subsection{device\_mode}\label{deviceux5fmode}

各控制器下各个设备的调光模式

\begin{longtabu}[c]{@{}lXXX@{}}
\toprule
字段名 & 类型 & 含义 & 备注\tabularnewline
\midrule
\endhead
controller\_addr & binary(8) & & PK1\tabularnewline
device\_addr & binary(8) & & PK2\tabularnewline
mode & int unsigned & & 0: manual, 1: auto, 2: schedule\tabularnewline
\bottomrule
\end{longtabu}

\subsection{device\_status}\label{deviceux5fstatus}

\begin{longtabu}[c]{@{}lXXX@{}}
\toprule
字段名 & 类型 & 含义 & 备注\tabularnewline
\midrule
\endhead
device\_id & int unsigned & device.id &\tabularnewline
version\_software & binary(4) & 软件版本号 &
LC300:只用2字节\tabularnewline
version\_hardware & binary(4) & 硬件版本号 &
LC300:只用2字节\tabularnewline
manufacturer\_info & varchar(450) & 厂商信息文本 &
上海:厂商描述。LC300:终端设备产品信息,\textless{}=32字节,ASCII码。\tabularnewline
manufacture\_time & datetime & 出厂时间 & LC300:无\tabularnewline
type & binary(4) & 设备类型 & 上海4.4.1 LC300 4.6.1
\texttt{0x0E}\tabularnewline
sn & varchar(64) & 产品序列号 &\tabularnewline
input\_volt & double & 输入电压采样值 &\tabularnewline
input\_amp & double & 输入电流采样值 &\tabularnewline
output\_volt & double & 输出电压采样值 &\tabularnewline
output\_amp & double & 输出电流采样值 &\tabularnewline
active\_power & double & 有功功率采样值 &\tabularnewline
temperature & smallint & 温度采样值 & LC300:1字节\tabularnewline
temperature\_guard & binary(4) & 过温保护状态 过温保护参数 & LC300 4.6
\texttt{0x0A}\tabularnewline
uptime & int unsigned & 上电工作时间 &\tabularnewline
total\_uptime & int unsigned & 总工作时间 &\tabularnewline
electricity\_consumption & int unsigned & 消耗电量值 &\tabularnewline
transition\_duration & tinyint unsigned & 调光渐变时间 &\tabularnewline
brightness & tinyint unsigned & 当前亮度 &\tabularnewline
brightness\_min & tinyint unsigned & 最小亮度值 &\tabularnewline
brightness\_max & tinyint unsigned & 最大亮度值 &\tabularnewline
brightness\_min\_physical & tinyint unsigned & 物理最小亮度值
&\tabularnewline
brightness\_max\_physical & tinyint unsigned & 物理最大亮度值 &
LC300未定义\tabularnewline
brightness\_power\_on & tinyint unsigned & 上电亮度值 &\tabularnewline
brightness\_default & tinyint unsigned & 默认故障亮度值 &\tabularnewline
brightness\_coefficiency & tinyint unsigned & 调光系数 & 实际亮度 =
设置亮度 * 调光系数 / 100\tabularnewline
status & binary(1) & 设备状态 & 上海:4.5.1 LC300:4.6
\texttt{0x06}\tabularnewline
comm\_status & binary(1) & 通讯状态 & 上海:4.5.1\tabularnewline
lamp\_status & binary(1) & 灯具状态 & 上海:4.5.1\tabularnewline
sensor\_i & binary(4) & 光强传感器采样值 & 上海:光感亮度采样值,
LC300:终端传感器设备光强值\tabularnewline
sensor\_l & binary(4) & 光照传感器采样值 &
LC300:终端传感器设备光照值\tabularnewline
sensor\_h & binary(4) & 湿度传感器采样值 &\tabularnewline
sensor\_t & binary(4) & 车流量传感器采样值 &\tabularnewline
group\_mask & binary(32) & 分组掩码 & 一共256
bit,若灯属于第N个组,则bit(N)=1。上海:组地址是1-254。LC300:组地址是0-63\tabularnewline
\bottomrule
\end{longtabu}

\subsection{device\_status\_edit}\label{deviceux5fstatusux5fedit}

StarRiver Config 在此记录用户录入的设备初始信息。

\begin{longtabu}[c]{@{}lXXX@{}}
\toprule
字段名 & 类型 & 含义 & 备注\tabularnewline
\midrule
\endhead
device\_id & int unsigned & device.id &\tabularnewline
transition\_duration & tinyint unsigned & 调光渐变时间 &\tabularnewline
brightness\_min & tinyint unsigned & 最小亮度值 &\tabularnewline
brightness\_max & tinyint unsigned & 最大亮度值 &\tabularnewline
brightness\_min\_physical & tinyint unsigned & 物理最小亮度值
&\tabularnewline
brightness\_max\_physical & tinyint unsigned & 物理最大亮度值 &
LC300未定义\tabularnewline
brightness\_power\_on & tinyint unsigned & 上电亮度值 &\tabularnewline
brightness\_default & tinyint unsigned & 默认故障亮度值 &\tabularnewline
brightness\_coefficiency & tinyint unsigned & 调光系数 & 实际亮度 =
设置亮度 * 调光系数 / 100\tabularnewline
group\_mask & binary(32) & 分组掩码 & 一共256
bit,若灯属于第N个组,则bit(N)=1。上海:组地址是1-254。LC300:组地址是0-63。\tabularnewline
\bottomrule
\end{longtabu}

\subsection{device\_status\_changes}\label{deviceux5fstatusux5fchanges}

记录设备状态变化。

\begin{longtabu}[c]{@{}lXXX@{}}
\toprule
字段名 & 类型 & 含义 & 备注\tabularnewline
\midrule
\endhead
device\_id & int unsigned & device.id & PK1\tabularnewline
field & varchar(32) & 被修改的字段名 & PK2\tabularnewline
field\_value & varchar(255) & 变化后的字段值 &\tabularnewline
time & timest\_amp & 修改时间 & PK3\tabularnewline
\bottomrule
\end{longtabu}

目前记录如下三个状态(亦即 \texttt{field}
的合法取值)的变化,由device\_status表中名为log\_changes的trigger执行:

\begin{enumerate}
\def\labelenumi{\arabic{enumi}.}
\itemsep1pt\parskip0pt\parsep0pt
\item
  ``status'': \texttt{field\_value} 存储 \texttt{HEX(status)}
\item
  ``comm\_status'': \texttt{field\_value} 存储
  \texttt{HEX(comm\_status)}
\item
  ``lamp\_status'': \texttt{field\_value} 存储
  \texttt{HEX(lamp\_status)}
\end{enumerate}

\subsection{device\_status\_history*}\label{deviceux5fstatusux5fhistory}

以下三张表的字段定义大部分相同,分别记录不同粒度的状态量历史。

\begin{enumerate}
\def\labelenumi{\arabic{enumi}.}
\itemsep1pt\parskip0pt\parsep0pt
\item
  device\_status\_history
\item
  device\_status\_history\_hourly
\item
  device\_status\_history\_daily
\end{enumerate}

三张表共同部分的定义如下:

\begin{longtabu}[c]{@{}lXXX@{}}
\toprule
字段名 & 类型 & 含义 & 备注\tabularnewline
\midrule
\endhead
device\_id & int unsigned & device.id & PK1\tabularnewline
field & varchar(32) & 被修改的字段名 & PK2\tabularnewline
field\_value & double & 字段采样值 &\tabularnewline
time & timest\_amp & 修改时间 & PK3\tabularnewline
\bottomrule
\end{longtabu}

两张汇总表有额外记录汇总时段内峰值、谷值的字段。

\begin{enumerate}
\def\labelenumi{\arabic{enumi}.}
\itemsep1pt\parskip0pt\parsep0pt
\item
  device\_status\_history\_hourly
\item
  device\_status\_history\_daily
\end{enumerate}

峰谷值定义如下:

\begin{longtabu}[c]{@{}lXXX@{}}
\toprule
字段名 & 类型 & 含义 & 备注\tabularnewline
\midrule
\endhead
field\_max & double & 汇总时段内最大采样值 &
浮点数可能无法精确表示整数采样\tabularnewline
field\_min & double & 汇总时段内最小采样值 &
浮点数可能无法精确表示整数采样\tabularnewline
\bottomrule
\end{longtabu}

目前对 \texttt{device\_status} 的如下状态进行采样(亦即 \texttt{field}
的合法取值):

\begin{enumerate}
\def\labelenumi{\arabic{enumi}.}
\itemsep1pt\parskip0pt\parsep0pt
\item
  ``output\_amp''
\item
  ``output\_volt''
\item
  ``active\_power''
\item
  ``brightness''
\end{enumerate}

有三个event更新这些表:

\begin{enumerate}
\def\labelenumi{\arabic{enumi}.}
\itemsep1pt\parskip0pt\parsep0pt
\item
  每5分钟,对 \texttt{device\_status} 进行采样,记录设备的状态量。
\item
  每天,将前一天的采样归纳成每小时均值。删除3天前的采样。
\item
  每周,将前一周的采样归纳成每日均值。删除一周前的小时均值。
\end{enumerate}

每月执行的事件 \texttt{Delete outdated status history} 会将
\texttt{device\_status\_history\_daily} 中超过365天前的记录删除。

\subsection{device\_schedule}\label{deviceux5fschedule}

LCP-SH-D:该功能是在单灯控制器与集中控制器失去联系后采用的异常模式,如果在某一时间段内没有设置该亮度值的话,将采用默认故障亮度值显示。

\begin{quote}
\textbf{注意} 调光计划的总数不能超过7。
\end{quote}

LC300没有定义。

\begin{longtabu}[c]{@{}lXXX@{}}
\toprule
字段名 & 类型 & 含义 & 备注\tabularnewline
\midrule
\endhead
id & bigint unsigned & 自增ID &\tabularnewline
device\_id & int unsigned & device.id &\tabularnewline
item & varchar(32) & 自动亮度表的一项 & HH,mm,value\tabularnewline
\bottomrule
\end{longtabu}

\subsection{firmware}\label{firmware}

\begin{longtabu}[c]{@{}lXXX@{}}
\toprule
字段名 & 类型 & 含义 & 备注\tabularnewline
\midrule
\endhead
id & int unsigned & &\tabularnewline
content & mediumblob & 二进制内容 &\tabularnewline
hash & char(32) & md5(content) &\tabularnewline
name & varchar(64) & 固件包名称 & 终端用\tabularnewline
version & binary(4) & 固件版本号 & 终端用\tabularnewline
note & varchar(300) & & UTF8\tabularnewline
\bottomrule
\end{longtabu}

MediumBlob可以存储16MB大小的固件,MySQL服务器也已经设置:

\texttt{max\_allowed\_packet=16M}

\subsection{frontend\_map\_bmp}\label{frontendux5fmapux5fbmp}

\begin{longtabu}[c]{@{}lXXX@{}}
\toprule
字段名 & 类型 & 含义 & 备注\tabularnewline
\midrule
\endhead
id & tinyint unsigned & &
id=0保留给GIS地图,不用插入记录。静态图的id从1开始。\tabularnewline
name & varchar(64) & &\tabularnewline
display\_order & tinyint unsigned & 前台程序界面上显示地图的顺序 &
不宜重复\tabularnewline
\bottomrule
\end{longtabu}

\subsection{frontend\_device\_map}\label{frontendux5fdeviceux5fmap}

描述各种设备在BMP地图上的信息。

\begin{longtabu}[c]{@{}lXXX@{}}
\toprule
字段名 & 类型 & 含义 & 备注\tabularnewline
\midrule
\endhead
frontend\_map\_bmp\_id & int unsigned & frontend\_map\_bmp.id &
PK1\tabularnewline
device\_type & tinyint unsigned & 表明设备类型,如控制器、灯等 &
PK2\tabularnewline
device\_id & int unsigned & device.id or controller.id &
PK3\tabularnewline
pos\_x & int & 设备在地图上的横坐标 &\tabularnewline
pos\_y & int & 设备在地图上的纵坐标 &\tabularnewline
latitude & double & 矢量地图时的纬度 &\tabularnewline
longitude & int & 矢量地图时的经度 &\tabularnewline
icon\_id & tinyint unsigned & & 对应的图标文件必须存在\tabularnewline
\bottomrule
\end{longtabu}

\begin{quote}
frontend\_map\_bmp\_id为0表示GIS地图,这条记录的 \texttt{latitude} 和
\texttt{longitude}
有意义。大于0的表示frontend\_map\_bmp中定义的静态地图, \texttt{pos\_x}
和 \texttt{pos\_y} 有意义。
\end{quote}

\subsection{frontend\_group}\label{frontendux5fgroup}

对于LCP-SH-D和LC300协议:现在用户的组和灯的组是统一的,即用户的第一个组就是所有灯的第一个组;对于CSA来说,一个用户的组可能对应控制器1的网关ID1和控制器2的网关ID2,所以还要建一张表来指定用户组和控制器里网关ID的关系(有点像以前LC200)。

\begin{longtabu}[c]{@{}lXXX@{}}
\toprule
字段名 & 类型 & 含义 & 备注\tabularnewline
\midrule
\endhead
id & tinyint unsigned & &\tabularnewline
name & varchar(64) & &\tabularnewline
color & tinyint unsigned & & Config程序里作为选项\tabularnewline
\bottomrule
\end{longtabu}

\subsection{frontend\_group\_devices}\label{frontendux5fgroupux5fdevices}

描述灯属于哪个组。

\begin{longtabu}[c]{@{}lXXX@{}}
\toprule
字段名 & 类型 & 含义 & 备注\tabularnewline
\midrule
\endhead
frontend\_group\_id & tinyint unsigned & frontend\_group.id &
PK1\tabularnewline
device\_id & int unsigned & device.id & PK2\tabularnewline
device\_display\_order & int unsigned & 界面显示该组内的灯时的序号 &
不宜重复\tabularnewline
\bottomrule
\end{longtabu}

\subsection{frontend\_scene}\label{frontendux5fscene}

这是用户定义的场景。

\begin{longtabu}[c]{@{}lXXX@{}}
\toprule
字段名 & 类型 & 含义 & 备注\tabularnewline
\midrule
\endhead
id & int unsigned & &\tabularnewline
name & varchar(64) & &\tabularnewline
display\_order & int unsigned & & 不宜重复\tabularnewline
\bottomrule
\end{longtabu}

\subsection{frontend\_scene\_item}\label{frontendux5fsceneux5fitem}

这是用户定义的场景。

\begin{longtabu}[c]{@{}lXXX@{}}
\toprule
字段名 & 类型 & 含义 & 备注\tabularnewline
\midrule
\endhead
id & bigint unsigned & &\tabularnewline
scene\_id & int unsigned & & frontend\_scene.id\tabularnewline
device\_addr & binary(8) & &
可以是单灯地址,也可以是组地址\tabularnewline
mode & int unsigned & & 0: manual, 1: auto, 2: schedule\tabularnewline
value & int unsigned & & brightness OR id auto policy OR id
schedule\tabularnewline
\bottomrule
\end{longtabu}

\subsection{frontend\_controller\_devices}\label{frontendux5fcontrollerux5fdevices}

描述灯属于哪个控制器。

对于CSA的协议,灯是属于控制器下的某个网关ID,所以可能会增加一个字段表示灯属于控制器的哪个网关。

\begin{longtabu}[c]{@{}lXXX@{}}
\toprule
字段名 & 类型 & 含义 & 备注\tabularnewline
\midrule
\endhead
controller\_id & int unsigned & controller.id & PK1\tabularnewline
device\_id & int unsigned & device.id & PK2\tabularnewline
device\_display\_order & int unsigned & 界面显示该控制器下的灯时的序号
&\tabularnewline
port\_on\_controller & smallint unsigned & 灯在控制器的哪个口上 &
可能不需要\tabularnewline
\bottomrule
\end{longtabu}

\subsection{sensor}\label{sensor}

传感器信息。

\begin{longtabu}[c]{@{}lXXX@{}}
\toprule
字段名 & 类型 & 含义 & 备注\tabularnewline
\midrule
\endhead
id & int unsigned & &\tabularnewline
name & varchar(255) & &\tabularnewline
type & int unsigned & & 0: virtual, 1: brightness, 2:
traffic\tabularnewline
source & varchar(255) & & ``controller\_addr,port(1-4)'' for those
attached on a controller; ``ip,port,user,pass,sql'' for those available
in a database\tabularnewline
raw\_value & double & &\tabularnewline
normalized\_value & int & & mapped to range {[}0,100{]}\tabularnewline
normalize\_method & int & & predefined method id here. Method lies in
source code.\tabularnewline
\bottomrule
\end{longtabu}

\subsection{auto\_policy}\label{autoux5fpolicy}

自动调光策略。

\begin{longtabu}[c]{@{}lXXX@{}}
\toprule
字段名 & 类型 & 含义 & 备注\tabularnewline
\midrule
\endhead
id & int unsigned & &\tabularnewline
name & varchar(255) & &\tabularnewline
display\_order & int unsigned & &\tabularnewline
sensor\_id & int unsigned & sensor.id &
可以是实际存在的,也可以是虚拟的传感器\tabularnewline
\bottomrule
\end{longtabu}

\subsection{auto\_policy\_item}\label{autoux5fpolicyux5fitem}

自动调光策略单项。

\begin{longtabu}[c]{@{}lXXX@{}}
\toprule
字段名 & 类型 & 含义 & 备注\tabularnewline
\midrule
\endhead
id & bigint unsigned & &\tabularnewline
auto\_policy\_id & int unsigned & auto\_policy.id &\tabularnewline
input & bigint unsigned & & {[}0,100{]}\tabularnewline
output & bigint unsigned & & {[}0,100{]}\tabularnewline
type & bigint unsigned & & 0: calculated, 1: user defined\tabularnewline
\bottomrule
\end{longtabu}

对于每个 \texttt{auto\_policy} ,都应生成101个策略项,分别对应
\texttt{input}
为0到100的情形。其中若干项是用户设定的,其余是插值计算而得。

\subsection{time\_schedule}\label{timeux5fschedule}

调光时间表。

\begin{longtabu}[c]{@{}lXXX@{}}
\toprule
字段名 & 类型 & 含义 & 备注\tabularnewline
\midrule
\endhead
id & int unsigned & &\tabularnewline
name & varchar(255) & &\tabularnewline
display\_order & int unsigned & & 不应重复\tabularnewline
\bottomrule
\end{longtabu}

\subsection{time\_schedule\_item}\label{timeux5fscheduleux5fitem}

调光时间表的逐条记录。

\begin{longtabu}[c]{@{}lXXX@{}}
\toprule
字段名 & 类型 & 含义 & 备注\tabularnewline
\midrule
\endhead
id & bigint unsigned & &\tabularnewline
time\_schedule\_id & int unsigned & & time\_schedule.id\tabularnewline
name & varchar(255) & & MM, DD, MM, DD, HH:mm:ss,
brightness\tabularnewline
\bottomrule
\end{longtabu}

\subsection{task\_todo}\label{taskux5ftodo}

待处理的用户操作。

服务器进程每隔一段时间在这里取任务,分别处理。

\begin{longtabu}[c]{@{}lXXX@{}}
\toprule
字段名 & 类型 & 含义 & 备注\tabularnewline
\midrule
\endhead
id & bigint unsigned & 自增ID & 当流水号\tabularnewline
time & datetime & &\tabularnewline
user\_id & int unsigned & user.id &\tabularnewline
cmd & smallint unsigned & 操作类型 &\tabularnewline
param & varchar(1024) & 操作的参数 &\tabularnewline
hash & varchar(40) & 命令的MD5散列值 & md5(time + user\_id + cmd +
param) 作为查询条件供前段程序在 \texttt{task\_done}
中查询任务完成状态\tabularnewline
\bottomrule
\end{longtabu}

\begin{quote}
\textbf{注意}:这个表使用MyISAM引擎存储。自增的id字段如果使用InnoDB存储,计数器在内存里,数据库服务器一旦重启,又将从1开始计数,完成任务后向
\texttt{task\_done} 中插入记录会引发冲突导致失败。
\end{quote}

\subsection{task\_done}\label{taskux5fdone}

记录那些已完成的用户操作。

\texttt{task\_todo} 中的任务,处理完毕后,从 \texttt{task\_todo}
删除,将结果插入到表 \texttt{task\_done} ,保持 \texttt{task\_todo}
中字段信息不变。

除了 \texttt{task\_todo} 的字段,另有以下字段:

\begin{longtabu}[c]{@{}lXXX@{}}
\toprule
字段名 & 类型 & 含义 & 备注\tabularnewline
\midrule
\endhead
finished & timest\_amp & 完成时间 &
插入记录时由数据库自动生成\tabularnewline
return\_code & smallint & 返回代码 & 成功:0;失败:控制器返回的确认码
bitwise-OR;控制器未能返回结果(如超时):-2。\tabularnewline
message & varchar(255) & 错误描述信息 & UTF8\tabularnewline
\bottomrule
\end{longtabu}

有一个 event \texttt{Remove outdated tasks from TaskDone*}
定期清理创建早前的任务,以保证这张表不会无限膨胀下去。
