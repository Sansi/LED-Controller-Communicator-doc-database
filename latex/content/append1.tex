\section{附录:表TaskTodo的任务类型及参数}\label{ux9644ux5f55ux8868tasktodoux7684ux4efbux52a1ux7c7bux578bux53caux53c2ux6570}

参数之间用英文的逗号 \texttt{,} 隔开;项与项之间用
\texttt{\textbackslash{}n} 分开。

\begin{longtabu}[c]{@{}rXXX@{}}
\toprule
Id & Command & Parameters & 备注\tabularnewline
\midrule
\endhead
0 & 密钥更新 & &\tabularnewline
1 & 重启控制器 & 控制器地址(8字节MAC) &\tabularnewline
2 & 控制器时间同步 & 控制器地址 &
如果控制器地址(8字节MAC)为广播地址的话则所有控制器都要时间同步\tabularnewline
3 & 设置控制器调光模式 & 控制器地址,模式 &
模式=0远程调光,调光指令由中央控制器发送;模式=1自动调光,集中控制器根据预先设置的时控计划进入时控模式\tabularnewline
4 & 设置控制器时控调光计划 &
控制器地址,终端地址(8字节MAC),开始月,开始日,结束月,结束日,时,分,秒,调光模式,调光值\texttt{\textbackslash{}n}
& 调光模式=0表示后面是直接亮度值,=1表示后面是场景值\tabularnewline
5 & 清空控制器时控调光计划 & 控制器地址 &\tabularnewline
6 & 设置控制器自动调光计划 & &\tabularnewline
7 & 清空控制器自动调光计划 & 控制器地址 &\tabularnewline
8 & 下载手机号码到控制器 & &\tabularnewline
9 & 查询控制器手机号 & 控制器地址 &\tabularnewline
10 & 删除控制器中所有手机号 & 控制器地址 &\tabularnewline
11 & 查询控制器内短信数量 & 控制器地址 &\tabularnewline
12 & 查询控制器某条短信内容 & &\tabularnewline
13 & 删除控制器某条短信内容 & &\tabularnewline
14 & 查询控制器状态 & 控制器地址 & 上海:命令
\texttt{0x0080};LC300:是多条命令的组合。\tabularnewline
15 & 查询控制器时控调光计划 & 控制器地址 &\tabularnewline
16 & 查询控制器自动调光计划 & 控制器地址 &\tabularnewline
17 & 查询控制器日志 & 控制器地址 &\tabularnewline
18 & 升级控制器固件 & 控制器地址,FirmwareMD5 &
控制器进入固件升级模式,固件升级,退出固件升级模式\tabularnewline
19 & 设置控制器所有属性 & &\tabularnewline
20 & 初始化终端信息 & 控制器地址 &\tabularnewline
21 & 查询终端初始化信息 & & 上海没有\tabularnewline
22 & 删除终端信息 & 控制器地址 & 删除控制器下所有的终端\tabularnewline
23 & 终端调光 & 控制器地址,终端地址,亮度值\texttt{\textbackslash{}n} &
这个可能主要用于单灯调光。控制器地址(8字节MAC)为广播地址的话是向所有控制器发;终端地址(8字节MAC)是广播地址的话是向控制器下的所有终端发、终端地址(8字节MAC)是组地址的话是向控制器下这个组的终端发、终端地址(8字节MAC)也可以是单个终端的地址\tabularnewline
24 & 初始化终端亮度 &
终端地址,调光渐变时间,上电亮度值,最大亮度值,最小亮度值,故障亮度值,调光系数
& 25-30的批处理\tabularnewline
25 & 设置终端调光渐变时间 & 终端地址,调光渐变时间 &\tabularnewline
26 & 设置终端上电亮度值 & 终端地址,上电亮度值 &\tabularnewline
27 & 设置终端最大亮度值 & 终端地址,最大亮度值 &\tabularnewline
28 & 设置终端最小亮度值 & 终端地址,最小亮度值 &\tabularnewline
29 & 设置终端故障亮度值 & 终端地址,故障亮度值 &\tabularnewline
30 & 设置终端调光系数 & 终端地址,调光系数 &\tabularnewline
31 & 终端恢复出厂设置 & 终端地址 &\tabularnewline
32 & 终端组设置 & 终端地址,组号\texttt{\textbackslash{}n} &
LC300:组0不用,就用组1-63\tabularnewline
33 & 终端组删除 & 终端地址 & 删除终端下的所有组\tabularnewline
34 & 终端场景设置 & 终端地址,场景号,亮度值\texttt{\textbackslash{}n}
&\tabularnewline
35 & 终端场景删除 & 终端地址 & 删除终端下的所有场景\tabularnewline
36 & 设置终端自动调光计划 & &\tabularnewline
37 & 删除终端自动调光计划 & 终端地址 &\tabularnewline
38 & 设置终端信息上报条件 & &\tabularnewline
39 & 设置终端报警阈值条件 & &\tabularnewline
40 & 查询终端状态量终端地址 & 终端地址\texttt{\textbackslash{}n}
&\tabularnewline
41 & 查询终端自动调光计划 & 终端地址 &\tabularnewline
42 & 查询终端组号 & 终端地址 &\tabularnewline
43 & 查询终端场景 & 终端地址 &\tabularnewline
44 & 升级终端固件 & 终端地址,FirmwareMD5 &
StarRiver服务按流程走\tabularnewline
45 & 清零终端总工作时间 & 终端地址 &\tabularnewline
46 & 清零终端总用电量 & 终端地址 &\tabularnewline
47 & 设置终端过温保护参数 & &\tabularnewline
48 & 开启关闭终端过温保护 & & LCP-SH-D没有\tabularnewline
49 & 重启StarRiver Server & &
任务返回值只有0,重启失败时从原理上就无法再将这个任务加到TaskDone了,超时还未从TaskDone读到重启结果可以认为是失败。\tabularnewline
58 & 终端自动调光 &
控制器地址,终端地址,自动表ID\texttt{\textbackslash{}n} &\tabularnewline
59 & 终端时控调光 &
控制器地址,终端地址,时控表ID\texttt{\textbackslash{}n} &\tabularnewline
60 & 设置紧急调光模式 & 控制器地址,模式 &
0:终端故障亮度,1:最近亮度,2:预存时控计划。通信时复用
\texttt{0x000D} 指令,用 \texttt{0xA0}, \texttt{0xA1} 和 \texttt{0xA2}
表示上述模式。
注意全局模式应统一,这里参数加控制器地址仅为了方便客户端识别操作失败的控制器。\tabularnewline
\bottomrule
\end{longtabu}

以下任务类型不由StarRiver提交,尽在StarRiver Server中用于数据帧的识别。

\begin{longtabu}[c]{@{}rll@{}}
\toprule
Id & Command & 备注\tabularnewline
\midrule
\endhead
50 & DownloadControllerFirmwareBin &\tabularnewline
51 & ExitControllerFirmwareUpdateMode &\tabularnewline
52 & DownloadDeviceFirmwareMetadata &\tabularnewline
53 & DownloadDeviceFirmwareBin &\tabularnewline
54 & DownloadFirmwareToDevice &\tabularnewline
55 & ExitDeviceFirmwareUpdateMode &\tabularnewline
56 & KeepAlive &\tabularnewline
57 & Register &\tabularnewline
\bottomrule
\end{longtabu}
